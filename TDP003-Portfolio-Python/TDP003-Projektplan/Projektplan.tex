\documentclass{TDP003mall}
\usepackage{listings}
\usepackage{color}

\definecolor{keyword}{RGB}{168,27,166}
\definecolor{string}{RGB}{50,140,55}
\definecolor{number}{RGB}{153,105,0}
\definecolor{comment}{RGB}{160,161,167}


\lstset{frame=tb,
  language=Python,
  aboveskip=3mm,
  belowskip=3mm,
  showstringspaces=false,
  columns=flexible,
  basicstyle={\small\ttfamily},
  numbers=none,
  numberstyle=\tiny\color{number},
  keywordstyle=\color{keyword},
  commentstyle=\color{comment},
  stringstyle=\color{string},
  breaklines=true,
  breakatwhitespace=true,
  tabsize=3
}


\newcommand{\version}{Version 1.1}
\author{Anton Sköld, \href{mailto:antsk320@student.liu.se}{antsk320@student.liu.se}\\
William Utbult, \href{mailto:wilut499@student.liu.se}{wilut499@student.liu.se}}
\title{TDP003 - Projektplan}
\date{2017-09-27}
\rhead{Anton Sköld\\
William Utbult}



\begin{document}
\projectpage
\tableofcontents

\section{Revisionshistorik}
  \begin{table}[!h]
    \begin{tabularx}{\linewidth}{|l|X|c|}
    \hline
    Ver. & Revisionsbeskrivning & Datum \\
    \hline
    1.0 & Revisionshistorik, Introduktion, Organisationsplan, Metodik, Produkt, Risker \& Tidsplan tillagda. & 2017-09-21\\
    \hline
    1.1 & Presentation förfinad. Milstolpar tillagda. Veckoplan utökad. & 2017-09-27\\
    \hline
    \end{tabularx}
  \end{table}

\section{Introduktion}
  Som en del av kursen TDP003: Egna datormiljön (IP-programmet) har vi i uppdrag att skapa en webbaserad portfölj i Python.

  Portföljen skall vara gjord för uppvisning av andra projekt som skapas här på IP-programmet.

  Hemsidan består av länkar till projekt, inkluderande sökfunktioner över projekten, samt sortering efter använda tekniker (ramverk, språk...).

  Databas över projekten ska finnas i JSON-format med tillgång via API.

  Presentationen är oberoende av databasen (via API).

  Portföljen ska vara anpassat för framtida arbetsgivare, samt intresserade, där de kan hitta information om projekt som ägaren har utvecklat.

\section{Organisationsplan}
  Anton Sköld - Utvecklare och Huvudansvarig

  William Utbult - Utvecklare och Huvudansvarig

  Filip Strömbäck - Reporter

  Dylan Mäenpää - Handledare

\section{Metodik}
  Versionshantering med Git \& gemensam repo på IDAs GitLab [\url{https://gitlab.ida.liu.se/antsk320/TDP003-Portfolio}]

  Utveckling utförs på portabla Linux-diskar privat eller gemensamt vid LiU labbsalar.

  TeamViewer för parprogrammering på distans.

  Trello för tekniska anteckningar \& planering.

  Korrekturläsning sker efter hand. Utvecklarna granskar och testar varandras kod och notifierar förändringar till relevanta medarbetere.

  Kodgranskning utförs av Dylan Mäenpää och/eller Filip Strömbäck, antingen redovisat i person eller via en Git-commit över e-post.


\section{Produkt}
  Servern kommer köra Python3 \& (Flask + Jinja2) för formatering och presentation i HTML5 \& CSS3.

  Databasen i JSON-format ser ut som följande:
  \begin{lstlisting}
  [
  	{
  		"project_id": 1,

  		"project_name": "Portfolio",

  		"start_date": "2017-09-15",

  		"end_date": "YYYY-MM-DD",

  		"course_id": "TDP003",

  		"course_name": "Projekt: Egna datormiljön",

  		"techniques_used": ["HTML", "CSS", "Python"],

  		"description": "Short description",

  		"long_description": "Long description",

  		"image": "https://url",

  		"external_link": "https://url"
  	},
    {...}
  ]
  \end{lstlisting}

\subsection{Verktyg}
  \begin{itemize}
    \item{Python3}
    \begin{itemize}
      \item{Flask}
      \item{Jinja2}
    \end{itemize}
    \item{HTML5}
    \begin{itemize}
      \item{CSS3}
    \end{itemize}
    \item{JSON}
    \item{OpenShift}
    \item{Git}
    \begin{itemize}
      \item{GitLab (IDA)}
    \end{itemize}
    \item{Trello}
  \end{itemize}

\subsection{Krav}
  Se systemspecifikationsdokumentet:\\
  \url{http://www.ida.liu.se/~TDP003/current/projekt/dokument/systemspecifikation.pdf}



\section{Risker}
  \begin{tabular}{ l|l }
    \hline
    Risk & Åtgärd \\
    \hline
    \hline
    Bortglömda deadlines & Håll ordentligt koll på schemat, arbeta långt före. \\
    \hline
	  Sjukdom & Arbeta hemifrån om möjligt, om inte, försök vara kontaktbar. \\
    \hline
	  Git-repon blir kaos & Ta hjälp av någon mer erfaren av Git, så snart som möjligt. \\
    \hline
	  Hårdvaruproblem & Ha backups, samt alternativa sätt att utveckla. \\
    \hline
  \end{tabular}

\section{Tidsplan}
  \begin{tabular}{ l|l }
    \hline
    Vecka & Beskrivning \\
    \hline
    \hline
	   v.38 & Skapande av projektlplansutkast.\\ & Förbättring av den gemensamma installationsmanualen \\ & Arbete på datalagret.\\
    \hline
	   v.39 & Datalagret färdigställt/godkänt.\\ & Föreläsning om Flask och Jinja2.\\ & Projektplanen kompletteras. \\ & Påbörjande av presentationslagret\\
    \hline
	   v.40 & Fortsatt arbete på presentationslagret.\\ & Påbörjning av systemdokumentationen.\\
    \hline
	   v.41 & Tillgängliggöra Portfolio via OpenShift.\\ & Första versionen av systemdokumentationen skriven/inlämnad. \\
    \hline
	   v.42 & Systemdemonstration för andra grupper.\\ & Systemdokumentation korrigerad. \\ & Testdokumentation \& Reflektionsdokument skrivna/inlämnade.\\
    \hline
  \end{tabular}

\subsection{Milstolpar}
  Dessa datum är ungefärliga. Milstolparna får gärna nås innan dessa datum.

\subsubsection{September}
  \begin{tabular}{ l|l }
    \hline
    Planerad & Mål\\Klar\\
    \hline
    \hline
    2017-09-27 Onsdag & Datalagret redo för redovisning. Fullt funktionellt \& klarar alla testfall.\\2017-09-20 Onsdag\\
    \hline
    2017-09-29 Fredag & Datalagret godkänt av assistent.\\2017-09-25 Måndag\\
    \hline
  \end{tabular}

\subsubsection{Oktober}
  \begin{tabular}{ l|l }
    \hline
    Planerad & Mål\\Klar\\
    \hline
    \hline
    2017-10-02 Måndag &	HTML-mallarna färdiga och redo att användas med programlogik.\\-\\
    \hline
    2017-10-06 Måndag & Presentationslagret till största del färdigt. Fixa buggar\/polera funktioner.\\-\\
    \hline
    2017-10-10 Tisdag &	Portfolion fullt funktionell och redo för publicering via OpenShift.\\-\\
    \hline
    2017-10-10 Tisdag &	Systemdokumentationen till största del redo för inlämning.\\-\\
    \hline
    2017-10-12 Torsdag & Portfolion tillgänglig via OpenShift.\\-\\
    \hline
    2017-10-12 Torsdag &	Systemdokumentationen inlämnad.\\-\\
    \hline
  \end{tabular}
\subsection{Deadlines}
  Dessa deadlines kommer från kursen TDP003.
\subsubsection{September}
  2017-09-07 [Torsdag]:   Planeringsdokument inlämnat.

  2017-09-14 [Torsdag]:   Lo-Fi prototyp \& grundläggande installationsmanual inlämnad.

  2017-09-21 [Torsdag]:   Första utkast av projektplanen är inlämnat. Samt första versionen av den gemensamma installationsmanualen klar.

  2017-09-28 [Torsdag]:   Bidraget med icke-trivial förbättring av den gemensamma installationsmanualen eller de gemensamma testerna via Git. Eventuella brister i den gemensamma installationsmanualen är korrigerade. Eventuella brister i projektplanen korrigerade och inlämnade.

  2017-09-29  [Fredag]:   Datalagret godkänt av assistent.

\subsubsection{Oktober}
  2017-10-12  [Torsdag]:   Portfolion tillgänglig via OpenShift. Första versionen av systemdokumentationen inlämnad.

  2017-10-17  [Tisdag]:   (prel.) Redovisning: Systemdemonstration för andra grupper, grund till testdokumentet.

  2017-10-19 [Torsdag]:   Eventuella brister i systemdokumentationen korrigerade och inlämnade. Testdokumentation inlämnad. Individuellt reflektionsdokument inlämnat.

\end{document}
