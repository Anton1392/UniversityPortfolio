\documentclass{TDP003mall}



\newcommand{\version}{Version 1.1}
\author{Anton Sköld, \url{antsk320@student.liu.se}\\
  William Utbult, \url{wilut499@student.liu.se}}
\title{TDP003 - Testdokument}
\date{2017-10-19}
\rhead{Anton Sköld\\
William Utbult}



\begin{document}
\projectpage
\tableofcontents

\begin{center}
Testlog\\
  \begin{tabular}{ |l|l|r| }
    \hline
    Version & Resultat & Datum\\
    \hline\hline
    1.0 & 1.3 Testfall 2 ger Internal Server Error. Resten godkända & 2017-10-19\\\hline
    1.1 & Samtliga testfall godkända & 2017-10-19\\\hline
  \end{tabular}
\end{center}

\section{Krav för presentationslager}
\subsection{Förstasida med bilder}
“Förstasida med bilder”\\
\textbf{Testfall 1}:\\
Indata: Skriv in URLn till hemsidan. I lokalt fall 127.0.0.1:5000\\
Förväntat resultat: En startsida med bilder.\\
Resultat: En startsida med bilder.\\
Godkänt: \textbf{Ja}

\subsection{Söksida med tillräcklig information}
“Söksida som visar en lista över projekt med kort information om varje projekt och som gör det möjligt att sortera dessa, samt söka bland dem genom ett formulär på sidan.”\\
\textbf{Testfall 1}: \\
Indata: Skriv in URLn till söksidan. I lokalt fall 127.0.0.1:5000/list\\
Förväntat resultat: En söksida som behandlar en lista av projekt enligt kravet.\\
Resultat: En söksida som behandlar en lista av projekt enligt kravet.\\
Godkänt: \textbf{Ja}

\subsection{Projektsida med projektinfo}
“Projektsida som visar fullständig information om ett projekt. GET-variabel
för att ange projekt-id: id URL: /project/id - där id är projektets
nummer.”\\
\textbf{Testfall 1}: \\
Indata: Skriv in URLn till projektsidan. I lokalt fall 127.0.0.1:5000/project/1\\
Förväntat resultat: En projektsidan som visar fullständig information om ett projekt.\\
Resultat: En projektsidan som visar fullständig information om ett projekt.\\
Godkänt: \textbf{Ja}

\textbf{Testfall 2}:\\
Indata: Skriv in URLn till projektsidan, men med felaktigt ID. I lokalt fall 127.0.0.1:5000/project/awd\\
Förväntat resultat: En 404-sida.\\
Resultat: En 404-sida.\\
Godkänt: \textbf{Ja}


\subsection{Tekniksida}
“Tekniksida som visar information om alla projekt utifrån använda tekniker.
URL: /techniques”\\
\textbf{Testfall 1}:\\
Indata: Skriv URLn till tekniksidan. I lokalt fall 127.0.0.1:5000/techniques.\\
Förväntat resultat: En tekniksida som visar samtliga tekniker som används i projekten.\\
Resultat: En tekniksida som visar samtliga tekniker som används i projekten.\\
Godkänt: \textbf{Ja}


\subsection{Projektbild på söksidan}
“För varje projekt ska en liten bild visas på söksidan och en stor på projektsidan. Det behöver inte vara samma bild. Bildtext för varje bild skall finnas.”\\
\textbf{Testfall 1}\\
Indata: Skriv URLn till söksidan. I lokalt fall 127.0.0.1:5000/list\\
Förväntat resultat: En liten bild till vardera projekt (om en giltig bildlänk finns för det projektet i databasen)\\
Resultat: En liten bild till vardera projekt (om en giltig bildlänk finns för det projektet i databasen)\\
Godkänt: \textbf{Ja}

\textbf{Testfall 2}:\\
Indata: Skriv URLn till projektsidan. I lokalt fall 127.0.0.1:5000/project/1\\
Förväntat resultat: En stor bild till vardera projekt (om en giltig bildlänk finns för det projektet i databasen)\\
Resultat: En stor bild till vardera projekt (om en giltig bildlänk finns för det projektet i databasen)\\
Godkänt: \textbf{Ja}

\subsection{Felmeddelanden}
“Vid fel ska systemet skriva ut informativa meddelanden till användaren
på en lämplig nivå för en slutanvändare. (Det vill säga, systemet ska
fånga och omvandla felkoder och statuskoder till begripliga meddelanden.)”\\
\textbf{Testfall 1}:\\
Indata: Gå in på ogiltig sökväg. I lokalt fall 127.0.0.1:5000/ogiltigt\\
Förväntat resultat: En sida med 404-information.\\
Resultat: En sida med 404-information\\
Godkänt: \textbf{Ja}


\subsection{Icke-existerande projekt}
“När en användare försöker visa ett projekt som inte finns, ska korrekt
statuskod returneras (dvs. 404).”\\
\textbf{Testfall 1}: \\
Indata: Skriv in URLn till projektsidan, med ett ID som inte existerar. I lokalt fall 127.0.0.1:5000/project/9001\\
Förväntat resultat: En sida med 404-information.\\
Resultat: En sida med 404-information\\
Godkänt: \textbf{Ja}


\section{Krav för datalager}
\subsection{Hanterad information.}
“Systemet ska kunna hantera följande information om ett projekt: projektnamn, projekt-id-nummer, startdatum, slutdatum, kurskod, kursnamn, kurspoäng, använda tekniker, kort beskrivning, lång beskrivning, liten och stor bild, gruppstorlek och en länk projektsida. Projektnamn och projekt-id är obligatoriska, övriga fält kan lämnas tomma.”\\
\textbf{Testfall 1}:\\
Indata: Öppna data.json-filen och kontrollera att vardera projekt har de fält specificerade av kraven.\\
Förväntat resultat: En .json-fil med projekt innehållande informationen specificerade ovan.\\
Resultat: En .json-fil med projekt innehållande informationen specifierade ovan.\\
Godkänt: \textbf{Ja}


\subsection{Projekt-id}
“Projekt-id ska vara ett unikt heltal för varje projekt”\\
\textbf{Testfall 1}:\\
Indata: Öppna data.json-filen och kontrollera att vardera projekt har ett unikt project-id, och att det är ett heltal och inte någon annan datatyp.\\
Förväntat resultat: En .json-fil med projekt där vardera projekt har ett unikt project-id specificerat med ett heltal.\\
Resultat: En .json-fil med projekt där vardera projekt har ett unikt project-id specificerat med ett heltal.\\
Godkänt: \textbf{Ja}


\subsection{Sekvens av tekniker}
“Varje projekt kan ha en sekvens av tekniker angivna”\\
\textbf{Testfall 1}:\\
Indata: Öppna data.json-filen och kontrollera att vardera projekt har en lista med tekniker i fältet \verb|techniques_used|.\\
Förväntat resultat: En .json-fil innehållandes projekt där alla projekt har en lista över använda tekniker.\\
Resultat: En .json-fil innehållandes projekt där alla projekt har en lista över använda tekniker.\\
Godkänt: \textbf{Ja}


\subsection{Sökning på godtycklig information}
“Sökning ska kunna göras på godtycklig projektinformation. Sökning kan ske på flera fält samtidigt. Sortering ska kunna göras på ett fält, i stigande och fallande träffordning. Man ska kunna filtrera utifrån använda tekniker i sökningen. Observera att allt ska fungera tillsammans, så att man kan söka på ett sökord, filtrera till vissa tekniker och sortera söklistan i en viss ordning samtidigt.”\\
\textbf{Testfall 1}:\\
Indata: Sök på projekt där \verb|project_name| = "Portfolio"\\
Förväntat resultat: Projektet med ID 1 och namnet "Portfolio"\\
Resultat: Projektet med ID 1 och namnet "Portfolio"\\
Godkänt: \textbf{Ja}

\textbf{Testfall 2}:\\
Indata: Sök på projekt där \verb|project_name| = "Detta är inget projekt"\\
Förväntat resultat: Inga sökresultat.\\
Resultat: Inga sökresultat.\\
Godkänt: \textbf{Ja}


\subsection{INTE LÄNGRE KRAV}

\subsection{INTE LÄNGRE KRAV}

\subsection{Datalagring}
“Data lagras i JavaScript Object Notation (JSON) i filen data.json. Filen ska lagras med UTF-8 teckenkodning.”\\
\textbf{Testfall 1}:\\
Indata: Kontrollera att data.json-filen har filavslutet .json och kan laddas in av datalagret.\\
Förväntat resultat: En inladdningsbar data.json-fil.\\
Resultat: En inladdningsbar data.json-fil.\\
Godkänt: \textbf{Ja}

\textbf{Testfall 2}: \\
Indata: Döp om projektet med ID = 1 till "Pårtföliä". Sök sedan efter ett projekt med namnet "Pårtföliä". Detta kontrollerar att .json-filen använder UTF-8 teckenkodning.\\
Förväntat resultat: Ett returnerat projekt som heter "Pårtföliä".\\
Resultat: Ett returnerat projekt som heter "Pårtföliä".\\
Godkänt: \textbf{Ja}


\subsection{Addition av data}
“Data läggs till i JSON-filer manuellt (eller av andra verktyg) i systemet.”\\
\textbf{Testfall 1}:\\
Indata: Lägg till en nytt projekt i data.json-filen. Vad fälten har för innehåll spelar ingen roll. Sök därefter på det projektet.\\
Förväntat resultat: Det tillagda projektet.\\
Resultat: Det tillagda projektet.\\
Godkänt: \textbf{Ja}

\textbf{Testfall 2}: \\
Indata: Lägg till ett projekt med felaktigt innehåll i data.json-filen och sök på det projektet.\\
Förväntat resultat: Sidan krashar med ett fel.\\
Resultat: Sidan krashar med ett fel.\\
Godkänt: \textbf{Ja}


\subsection{Live-data}
“Förändring av data.json ska slå igenom direkt i systemet utan omstart\\
av webbserver.”\\
\textbf{Testfall 1}: \\
Indata: Starta sidan. Gå in på /project/1 och notera projektbeskrivningen. Gå sedan in i data.json-filen och ändra beskrivningen på projektet med \verb|project_id| = 1. Uppdatera sedan projektsidan.\\
Förväntat resultat: En projektsida med uppdaterad beskrivning.\\
Resultat: En projektsida med uppdaterad beskrivning.\\
Godkänt: \textbf{}


\subsection{Dataredigering}
“(Frivilligt) Utvidga systemet med en administrativ sida för redigering\\
av data.”\\
\textbf{Testfall 1}:\\
Indata: \\
Förväntat resultat:\\
Resultat:\\
Godkänt: \textbf{?}

\textbf{Testfall 2}:\\
Indata:\\
Förväntat resultat:\\
Resultat:\\
Godkänt: \textbf{}


\end{document}
